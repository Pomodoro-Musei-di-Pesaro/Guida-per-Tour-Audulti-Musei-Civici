\documentclass[hidelinks,12pt,a4paper]{article}
\usepackage[italian]{babel}
\usepackage[utf8]{inputenc}
\usepackage{fourier} 

% To avoid GitHub Action error
\usepackage{hyperref}

% manage minipage
\usepackage{tabularx}

% Stop hyphenation
\usepackage[none]{hyphenat}

% Images
\usepackage{graphicx}
\usepackage{caption}
\usepackage{subcaption}
\usepackage{float}
\graphicspath{ {../Images} }

% Remove first empty page
\usepackage{atbegshi}
\AtBeginDocument{\AtBeginShipoutNext{\AtBeginShipoutDiscard}}

% Justifying text
\emergencystretch 3em

% Adjust paragraph.
\usepackage{changepage}

% Coloring links
\usepackage{xcolor}

%Command to zoom in --- This isn't the right algorithm to do that.
\usepackage{mwe}
\makeatletter
\newsavebox\zb@x
\newcounter{z@@m}
\usepackage{calc}
\newdimen\B@r\newdimen\P@r
\newdimen\@zw\newdimen\@zh\newdimen\@zd

\newcommand{\zoombox}[2][0]{%
	\leavevmode%
	\sbox\zb@x{#2}%
	\setlength\B@r{1pt*\ratio{\wd\zb@x}{\ht\zb@x+\dp\zb@x}}%
	\setlength\P@r{1pt*\ratio{\paperwidth}{\paperheight}}%
	\ifdim\B@r>\P@r\relax%
	\setlength\@zw{\wd\zb@x}\setlength\@zh{\@zw*\ratio{\paperheight}{\paperwidth}}%
	\setlength\@zd{(\@zh-\ht\zb@x-\dp\zb@x)*\real{0.5}+\dp\zb@x}%
	\setlength\@zh{\@zh-\@zd}%
	\else%
	\setlength\@zh{\ht\zb@x+\dp\zb@x}%
	\setlength\@zw{\@zh*\ratio{\paperwidth}{\paperheight}}%
	\setlength\@zh{\ht\zb@x}\setlength\@zd{\dp\zb@x}%
	\fi%
	\makebox[0pt][l]{\makebox[\wd\zb@x][c]{\makebox[\@zw][l]{%
				\pdfdest name {zbfs\thez@@m} fitr
				width  \@zw\space
				height \@zh\space
				depth  \@zd\space
	}}}%
	\pdfdest name {zb\thez@@m} fitr
	width  \wd\zb@x\space
	height \ht\zb@x\space
	depth  \dp\zb@x\space
	\immediate\pdfannot 
	width  \wd\zb@x\space
	height \ht\zb@x\space
	depth  \dp\zb@x\space
	{%
		/Subtype/Link/H/N
		/Border [0 0 #1 [1 2]]
		/A <<
		/S/JavaScript
		/JS (
		if(typeof(zoomed)=='undefined'||!zoomed){
			var lastView=this.viewState;
			if(app.fs.isFullScreen) this.gotoNamedDest('zbfs\thez@@m');
			else this.gotoNamedDest('zb\thez@@m');
			zoomed=true;
		}else{
			this.viewState=lastView;
			zoomed=false;
		}
		)
		>>
	}%
	\usebox{\zb@x}%
	\stepcounter{z@@m}%
} 
\makeatother

% License
\usepackage[
type={CC},
modifier={by-nc-sa},
version={4.0},
]{doclicense}

\begin{document}
	\begin{flushleft}
		
		\title{\textbf{\centering{Manuale per operatori}\\Tour guidato per adulti ai Musei Civici}}
		\author{Francesco Rombaldoni}
		\date{}
		
		\maketitle
		\begin{adjustwidth}{-30mm}{-30mm}
			\vspace*{\fill}
			\centering
			\fboxrule=2pt
			\fbox
			{
				\begin{minipage}{0.85\linewidth}
					Il seguente documento è ottimizzato per la visualizzazione digitale con \href{https://get.adobe.com/it/reader/}{\textcolor{blue}{Adobe~Acrobat~Reader}}.  
				\end{minipage}
			}
		\end{adjustwidth}
		
		% Adjust page counter
		\setcounter{page}{1}
		\newpage
		\topskip0pt
		\vspace*{\fill}
		Questa documentazione è un supporto volto a formare e facilitare gli Operatori Museali, nel far comprendere alle persone le opere visibili all'interno dei Musei Civici, attraverso un Tour guidato da noi appositamente elaborato. 
		\vspace*{\fill}
		\newpage
		\tableofcontents
		\newpage
		
		\section{Scalone d'ingresso}
		
		\subsection{La Medusa di Ferruccio Mengaroni}
		L'imponente tondo in rilievo realizzato in ceramica è ispirato alla mitologica figura greca della Medusa, raffigurata nell'attimo in cui viene decapitata da Perseo, ma rappresenta anche l’autoritratto dell'artista.\\
		Si dice che l'opera, fin dagli inizi della sua realizzazione, fosse avvolta da una maledizione; infatti, mentre l'artista si stava ritraendo, lo specchio dal quale si stava osservando si ruppe.
		Non pensate anche voi visitatori che sia ironico, se si considera che Perseo utilizzò lo scudo proprio a mo' di specchio per sconfiggere la Medusa, che sia stato proprio uno specchio rotto a gettare sciagura sul ceramista che la realizzò?\\
		Il 13 maggio 1925 l'opera fu trasportata per la seconda Biennale d'Arte a Monza ma, durante il trasporto lungo la ripida scalinata, il tondo si sbilanciò. Mengaroni, colto dal terrore di perdere il suo capolavoro, corse incontro ad esso per sorreggerlo, ma ne venne schiacciato e morì.\\
		Mengaroni è una delle più importanti figure nell'ambito della ceramica tra la seconda metà dell'Ottocento e gli inizi del Novecento; il suo stile è caratterizzato dal "revival" nei confronti delle opere rinascimentali, tipico dell'epoca.
		La sua formazione avvenne attraverso lo studio delle collezioni ceramiche presenti nei Musei Civici e la loro riproduzione, per poi creare opere proprie attraverso lo stile dell'istoriato e delle grottesche.
		
		\vspace*{0.2cm}
		
		\begin{minipage}{\linewidth}
			\vspace*{0.2cm}
			\centering
			\zoombox{\includegraphics[scale=1.5]{Mengaroni_Ferruccio-Medusa.jpg}}
			\captionof{figure}{Mengaroni Ferruccio - Medusa.}
		\end{minipage}
		
		\section{Sala Giovanni Bellini}
		In questa sala sono presenti principalmente dipinti derivanti dall'acquisizione delle opere all'interno delle chiese del pesarese, oggetto della soppressione delle congregazioni religiose promossa dal Regno d'Italia nell'anno 1861.\\
		Le opere appartengono al periodo che va dal Trecento al Quattrocento, ovvero tra il periodo gotico-medievale e il Rinascimento.
		Sono state realizzate in scuole attive in ambito veneto, fiorentino e locale, alle quali i Malatesta (Urbino) e gli Sforza (Milano) commissionavano opere per arricchire la loro collezione artistica.\\
		Altre opere qui esposte fanno parte della collezione Ercolani-Rossini.
		
		\subsection{Incoronazione della Vergine di Giovanni Bellini}
		La celebre Pala d'altare fu realizzata a Venezia nell'anno 1475 e successivamente trasportata via mare (smontando e riassemblando l'opera in loco) per la Chiesa di S. Francesco, meglio nota nel pesarese come Madonna delle Grazie; fu probabilmente commissionata dagli Sforza, ma non vi è certezza poiché non sono stati ritrovati i documenti originali.\\
		È senza dubbio uno dei maggiori capolavori appartenenti al Rinascimento in Italia.\\
		Una delle particolarità di quest'opera è che fu ideata come le scatole cinesi, cioè come un’opera nell'opera. Infatti, la si può leggere sia orizzontalmente che dall'alto verso il basso.
		La cornice contribuisce a rendere appieno la storia dell'Incoronazione di Maria da parte di Gesù Cristo, attorniata da quattro Santi, rispettivamente: S. Girolamo, S. Marco, S. Giorgio e S. Terenzio (Patrono di Pesaro).\\
		Nella parte superiore della Pala, detta Cimasa, Napoleone trafugò parte del dipinto per poterne conservare il ricordo. Successivamente, grazie all'intervento dello scultore Antonio Canova, furono restituite all'Italia varie di queste opere trafugate, le quali sono ora conservate nei Musei Vaticani.\\
		L'opera fu oggetto di vari restauri, tra i quali quello del 2008 in preparazione per l'esposizione in una mostra all'interno delle Scuderie del Quirinale. Per la speciale occasione venne riunita alla Cimasa la parte dove veniva raffigurata l’Imbalsamazione di Cristo.\\
		Lo stile di Giovanni Bellini (artista la cui formazione avvenne nella bottega veneta del padre) è caratterizzato dal rendere i personaggi sacri più realistici e umani che divini, attraverso la tecnica ad olio sviluppata dai Fiamminghi nelle Fiandre.
		\newpage
		
		\begin{minipage}{\linewidth}
			\centering
			\zoombox{\includegraphics[scale=0.1]{Bellini_Giovanni-Incoronazione_della_Vergine.jpg}}
			\captionof{figure}{Bellini Giovanni - Incoronazione della Vergine.}
		\end{minipage}
		
		\begin{minipage}{\linewidth}
			\centering
			\begin{minipage}{0.4\linewidth}
				\zoombox{\includegraphics[scale=0.65]{Pala_di_pesaro_incoronazione.jpg}}
				\captionof{figure}{\centering{Dettaglio: Incoronazione della Vergine.}}
			\end{minipage}
			\hfill
			\begin{minipage}{0.4\linewidth}
				\zoombox{\includegraphics[scale=0.1]{Pala_di_Pesaro_S.Giorgio_e_il_Drago.jpg}}
				\captionof{figure}{\centering{Dettaglio: S. Giorgio e il Drago.}}
			\end{minipage}
		\end{minipage}
		
		\subsection{Beata Michelina e Santi di Jacobello Del Fiore}
		Un'antica leggenda narra che la città di Pesaro è protetta dalla profezia della Beata Michelina che recita: "Proteggerò la mia città. Essa tremerà ma non cadrà”.\\
		Beata Michelina nacque a Pesaro nel 1300 da una ricca e nobile famiglia originaria di Farneto, l’antica famiglia "Deutaleve", che nel XV secolo prese il nome di Metelli.
		Dopo aver ricevuto una formazione adeguata al suo stato e aver sposato un nobile (da alcuni ritenuto appartenente alla famiglia dei Malatesta), rimase vedova a vent'anni e, benché ricca e avvenente, non volle più risposarsi.\\
		Michelina, già molto religiosa, durante l'incontro con la pellegrina Soriana (o Sira), ospite presso la sua famiglia, venne colpita dalla sua bontà d'animo, dalla religiosità e dal distacco dalla vita mondana.
		Ciò la indusse, dopo la morte del figlio, ad abbandonare le sue ricchezze e a intraprendere il cammino della fede con rinnovato ardore.
		Decise quindi di diventare “terziaria” francescana, dispensando i suoi averi in favore dei poveri, edificando chiese, riscattando la libertà di carcerati pagandone i debiti, creando doti per fanciulle ed orfani.\\
		Si ridusse così in povertà e visse di elemosine e penitenza (autoflagellandosi durante le processioni).
		Nel 1347, assieme al Beato Cecco, fondò la Confraternita dell'Annunziata, alla quale donò la sua casa con lo scopo di assistere e seppellire gli infermi.
		Si narra che in vita, con le sue preghiere, riuscì a evitare catastrofi e morti.\\
		Il 19 giugno dell'anno 1356, consumata dai digiuni e dalle sofferenze, si ammalò gravemente e, dopo aver giurato di proteggere Pesaro, morì.
		La salma, rinvenuta presso un vicolo di fronte alla Chiesa di S. Cassiano (oggi denominata Via Michelina Metelli), venne trasferita presso la Chiesa di S. Francesco (ora Madonna delle Grazie).\\
		Questo polittico ligneo, in cui la Santa è ritratta al centro emergendo con forza plastica, attorniata da archetti a sesto acuto dal gusto tardo gotico (prospero nell'area adriatica agli inizi del Quattrocento), venne realizzato dal veneziano Jacobello Del Fiore per la Cappella della Michelina. L'opera fu fortemente voluta da Pandolfo II Malatesta, che fece realizzare anche un'urna in marmo nel 1708 per la Santa, poiché si salvò da un naufragio ricorrendo al suo aiuto in preghiera, sciogliendo così il voto fatto.\\
		Si narra che la stessa Santa scampò miracolosamente a un naufragio in Palestina.
		
		\begin{minipage}{\linewidth}
			\centering
			\vspace*{0.2cm}
			\zoombox{\includegraphics[scale=0.5]{Jacobello_Del_Fiore-Beata_Michelina_e_Santi.jpg}}
			\captionof{figure}{Beata Michelina e Santi di Jacobello Del Fiore}
		\end{minipage}
		
		\subsection{S. Terenzio di Antonio Bellinzoni da Pesaro}
		Il dipinto costituiva il coperchio del sarcofago di S. Terenzio, come testimonia l’antico manoscritto incollato nella parte sinistra dello stesso.\\
		San Terenzio è rappresentato (in linea con l'iconografia del Quattrocento) come un nobile giovane che regge in una mano un libro e nell'altra la palma del martirio.
		In alternativa poteva essere raffigurato nelle vesti di un soldato romano, come nella Pala di Giovanni Bellini.
		Diversamente, l’iconografia medievale lo rappresentava come vescovo.\\
		In quest'opera l'eleganza della linea rivela ancora un influsso tardo gotico, che però si sta aprendo alla novità della ricerca volumetrica e della geometria dell'area fiorentina.\\
		L'artista pesarese Antonio Bellinzoni (trasferitosi a Pesaro nel 1410, pare da Parma) ebbe la sua formazione in Emilia presso gli ultimi maestri del tardo gotico.
		Influenzato dallo stile naturalistico padano usato nella bottega del padre, operò prevalentemente nelle Marche, dove vennero rinvenute svariate sue opere.
		
		\begin{minipage}{\linewidth}
			\centering
			\vspace*{0.2cm}
			\zoombox{\includegraphics[scale=0.2]{Bellinzoni_Giovanni_Antonio_da_Pesaro-San_Terenzio.jpg}}
			\captionof{figure}{San Terenzio}
		\end{minipage}
		
		\subsection{Sogno della Vergine di Michele di Matteo}
		L'opera, realizzata da Michele di Matteo nel 1410 (pittore di origini probabilmente bolognesi che operò soprattutto in Italia centro-settentrionale) e appartenente alla Collezione Ercolani Rossini, raffigura, su di un pregiato fondale decorato con foglie d'oro, la Vergine addormentata intenta a sognare la tragica "Cacciata dal Paradiso di Adamo ed Eva".\\
		Secondo Roberto Longhi (celebre critico d'arte), l'opera è caratterizzata da uno stile delicato e allo stesso tempo pungente, visibile soprattutto dalle aureole punzonate poste sui capi dei personaggi.\\
		L'opera ebbe un discreto successo nel Trecento; venne notata dal celebre Rossini, il quale amava le decorazioni in oro, la rappresentazione di gioielli all'interno dell'opera e il colore rosso vermiglio presente nei panneggi (il colore era ottenuto dalla particolare lavorazione di alcuni insetti, detti "vermi", per questo motivo si chiama "vermiglio").
		
		\begin{minipage}{\linewidth}
			\vspace*{0.2cm}
			\centering
			\zoombox{\includegraphics[scale=0.4]{Michele_di_Matteo-Sogno_della_Vergine.jpg}}
			\captionof{figure}{Sogno della Vergine}
		\end{minipage}
		
		\section{Sala delle ceramiche della collezione di Domenico Mazza}
		Questa sala ospita opere di varia natura e di ampia fascia temporale, provenienti da collezioni private, di cui la più importante è quella di Domenico Mazza (originario di Urbino).\\ 
		Le ceramiche sono decorate con la raffinata tecnica della maiolica, tipica dello stile rinascimentale del Quattrocento nel territorio del pesarese, di Urbania (chiamata pure Casteldurante) e di Urbino; si possono notare le ceramiche con decoro floreale (alla rosa) tipico della maiolica pesarese, il cui stile ebbe un forte "revival" nel Settecento con il recupero delle arti applicate.\\
		Tutt'ora l'arte della ceramica è una delle eccellenze artistiche tipiche del pesarese.
		
		\subsection{Caccia al Cinghiale Calidonio della bottega di Lanfranco dalle Gabicce}
		Questa coppa in ceramica ritrae il mitologico episodio di Meleagro intento nella caccia del cinghiale Calidonio.\\
		Da notare il dinamismo dalla pennellata veloce ed energica a confronto con impervie ed enormi montagne ed improbabili edifici.\\
		Questa coppa è caratterizzata da colori accesi, una linea "nervosa" e dinamica che rende perfettamente la velocità della scena rappresentata e ci lascia assaporare il suo gusto tipicamente manierista.\\
		Nel 1541 Guidobaldo II della Rovere diede nuova importanza alla lavorazione della ceramica e fece rinascere questa arte nel Cinquecento a Urbino.
		
		\begin{minipage}{\linewidth}
			\vspace*{0.2cm}
			\centering
			\zoombox{\includegraphics[scale=0.5]{Lanfranco_Girolamo_dalle_Gabicce-caccia_al_cinghiale_calidonio.jpg}}
			\captionof{figure}{Caccia al cinghiale calidonio}
		\end{minipage}
		
		\subsection{Ercole, Anteo e Caco steso a terra della bottega di Fontana da Urbino}
		Il piatto rappresenta Ercole, muscoloso e possente, i cui nemici sono ormai esanimi.\\ 
		Questo tipo di illustrazione è stato ispirato dalle opere di Raffaello presenti nei Musei Vaticani (Domus Aurea), per questo motivo lo stile prende il nome di Raffaellesco-Grottesco.
		Nel XVI secolo lo stile Raffaellesco prese il sopravvento sullo stile Istoriato, tipico del Cinquecento a Urbino.\\
		
		\begin{minipage}{\linewidth}
			\centering
			\vspace*{0.2cm}
			\zoombox{\includegraphics[scale=1.5]{Ercole_e_Anteo_Caco_a_terra_senza_vita.jpg}}
			\captionof{figure}{Ercole e Anteo Caco a terra senza vita}
		\end{minipage}
		
		\subsection{Faustina Bella, Coppa con profilo di donna della bottega di Casteldurante (Urbania)}
		Il volto marmoreo cinto da un elmo, il busto attorniato da una ricca ghirlanda e i colori che permettono al soggetto di emergere dallo sfondo blu, fanno parte della serie "Belle Donne", che in questo caso particolare ritrae Faustina (opera del 1522).\\
		È stato il primo esemplare datato facente parte dei cosiddetti “Doni Amatori” (ovvero doni di fidanzamento o nozze per la signora), nei quali era solito rappresentare donne i cui volti avevano poche varianti, con l'aggiunta dell'aggettivo "bella" o altri complimenti riguardanti le doti della consorte.
		Questi erano doni di corteggiamento (considerati eleganti e di pregio) che i nobiluomini facevano alla donna amata o voluta per interessi di famiglia.
		
		\begin{minipage}{\linewidth}
			\centering
			\vspace*{0.2cm}
			\zoombox{\includegraphics[scale=0.5]{Faustina-Pittore_in_Casteldurante.jpg}}
			\captionof{figure}{Ercole e Anteo Caco a terra senza vita}
		\end{minipage}
		
		\subsection{San Giuda Taddeo di Nicola da Urbino}
		L'iconografia deriva da un’incisione di Marcantonio Raimondi ispirata a un disegno di Raffaello.\\
		Il personaggio rappresentato è San Giuda Taddeo, la cui identificazione è definita dall'alabarda (lo strumento del suo martirio).\\
		Lo stile di Nicola da Urbino è caratterizzato da finezza, grazia, ariosità dei panneggi e dall'intensità contrapposta alla dolcezza dello sguardo.\\
		Il suo operato è stato attivo dal 1520 al 1538.\\
		La firma sul retro è dell'illustratore Mastro Giorgio Andreoli da Gubbio; la sua cittadina faceva parte dei possedimenti del ducato di Urbino, che era uno dei grandi centri di produzione delle ceramiche.
		
		\subsection{San Girolamo Penitente di Mastro Giorgio da Gubbio (1522)}
		Il vassoio umbonato (che ricorda come forma la borchia degli scudi, oltre al fatto che è stato usato come riferimento per nominare il cappello di alcuni tipi di funghi) articola la scena in due spazi; l’iconografia del Santo in adorazione del Crocifisso è collocata nello spazio più esterno del vassoio, mentre al centro è rappresentato l’attributo del Santo, cioè il leone.\\
		La particolare lavorazione di questa ceramica è stata eseguita con l’antica tecnica del lustro, attribuibile al IX-X secolo d.C., che attraverso l’applicazione di speciali impasti d'ossido d'argento e rame e una complessa tecnica di cottura rende possibile ottenere sfumature particolarmente luminose di color oro o rosso rubino.\\
		Questa complessa lavorazione venne ripresa nel XV secolo a Gubbio grazie all'incentivo del ducato urbinate.
		
		\begin{minipage}{\linewidth}
			\centering
			\vspace*{0.2cm}
			\zoombox{\includegraphics[scale=1.2]{Pittore_del_Giudizio_di_Paride-san_Girolamo_penitente.jpg}}
			\captionof{figure}{Pittore del Giudizio di Paride - San Girolamo Penitente}
		\end{minipage}
		
		\subsection{Adorazione dei Pastori di Francesco Xanto Avelli}
		Francesco Xanto Avelli rappresenta attraverso questa maiolica rinascimentale (che trae spunto da un disegno del celebre Parmigianino) un gruppo di pastori che converge verso la Vergine seduta, ovvero verso il centro focale della rappresentazione.\\
		La maiolica rinascimentale si differenzia dagli altri generi per l'uso di colori come il blu, il verde, l’arancione e il marrone, i quali sono usati in gran misura nella composizione.\\
		Si può notare che al basamento della colonna è riportata la data di realizzazione (1537) e sul gradino sottostante è presente la sigla dell'autore.\\
		Francesco Xanto Avelli è originario di Rovigo (anni Venti del Cinquecento); fu allievo di Nicola da Urbino, che lo descrive come un personaggio eclettico e singolare.
		Si dice che compose persino un’opera in versi dedicata al duca Francesco Maria I della Rovere.\\
		Frequentemente traeva spunto da stampe dei grandi maestri del Rinascimento: Raffaello, Taddeo Zuccari e Battista Franco, inserendo le suddette figure (com'era tipico dello stile Istoriato Roveresco) all'interno di paesaggi d'invenzione dell'artista.
		
		\begin{minipage}{\linewidth}
			\centering
			\vspace*{0.2cm}
			\zoombox{\includegraphics[scale=1.5]{Adorazione_dei_pastori.jpg}}
			\captionof{figure}{Adorazione dei pastori}
		\end{minipage}
		
		\section{Sala arredi e sculture della collezione Mosca}
		In questa sala è presente la collezione di oggetti ed arredi del periodo tra Ottocento e Novecento appartenenti alla marchesa Toschi Mosca, di cui possiamo ammirare il ritratto appena si entra all'interno del Museo, sulla destra, oltrepassando il punto di accettazione.
		La marchesa era un'amante del gusto antiquario e qui esposta in maniera permanente, come da lei voluto, possiamo ammirare la sua collezione di oggetti di uso domestico personale. L'intento della marchesa era infatti quello di creare un Museo d'arte industriale accessibile al pubblico a Pesaro, dove ella stessa viveva, per far sì che altri giovani artisti si potessero ispirare alle bellezze del passato per creare i loro capolavori.
		
		\subsection{Stipi con vedute di Roma di Johann Wilhelm Baur}
		In questa sala dedicata agli arredi della marchesa possiamo ammirare i pregiati Stipi con Vedute di Roma e ne possiamo notare l'usanza di decorare queste cassettiere di uso nobiliare con delle placche raffiguranti vedute dell'Urbe. Questa usanza è fatta risalire agli anni sessanta del Seicento e serviva ai nobili per far sfoggio dei loro costosi viaggi con gli ospiti.
		In questi stipi in particolare sono ritratte le stupende vedute di Roma, dipinte con l'utilizzo delle tempere su pergamena in maniera estremamente realistica e con minuzia di particolari dallo strasburghese Johann Wilhelm Baur, pittore del periodo barocco presente a Roma, che collaborò con Giacomo Herman, ebanista, tra il 1665 e il 1667 per la realizzazione degli stipi. Possiamo notare, osservandoli con attenzione, la firma dell'ebanista all'interno del mobile.
		
		\begin{minipage}{\linewidth}
			\vspace*{0.2cm}
			\begin{tabularx}{\linewidth}{XX}
				{
					\begin{minipage}{\linewidth}
						\zoombox{\includegraphics[scale=0.35]{Vedute_di_Roma79.jpg}}
						\captionof{figure}{Stipo con vedute di Roma.}
					\end{minipage}
				}&{
					\begin{minipage}{\linewidth}
						\hspace*{0.5cm}
						\zoombox{\includegraphics[scale=0.35]{Vedute_di_Roma80.jpg}}
						\captionof{figure}{Stipo con vedute di Roma.}
					\end{minipage}
				}
			\end{tabularx}
		\end{minipage}
		
		\subsection{Orologio Notturno di Andrea Nattan}
		L'Orologio Notturno è un oggetto di arredo domestico-funzionale che la marchesa custodiva nella propria camera da letto; le permetteva di leggere l'ora attraverso la luce di una candela che illuminava un forellino sorretto da un'illustrazione di un putto nel mobile (la candela veniva periodicamente sostituita dalla servitù). La marchesa si dilettava così guardando il bel mobile la sera, poiché soffriva d'insonnia, ed esso la aiutava a comprendere quando fosse giunto il momento di alzarsi o meno.
		L'oggetto d'arredo è una grande cassa in legno pregiato di epoca barocca (XVII secolo).
		La scritta sorretta in primo piano dall'angioletto recita: "Volat irreparabile Tempus", che significa "Il tempo vola inesorabilmente", ed era attorniato dall'allegoria delle quattro stagioni, sovrastate dalla figura di un uomo barbuto alato che rappresentava appunto il Tempo.\\
		Fanno parte inoltre della collezione della marchesa anche oggetti in madreperla, avorio, uno specchio di pregevole fattura in vetro soffiato di Murano (Venezia) con decorazioni incise in argento a motivo di uva, varie cornici di epoca barocca ed infine varie sculture.
		
		\begin{minipage}{\linewidth}
			\centering
			\zoombox{\includegraphics[scale=0.4]{Nattan_F.Andrea-orologio_notturno.jpg}}
			\captionof{figure}{\centering{Orologio~notturno.}}
		\end{minipage}
		
		\section{Sforza signori di Pesaro nella seconda metà del Quattrocento}
		Lo stemma con leone rampante e ramo di cotogno (mela cotogna, ramo di famiglia), tipico della casata, è presente sulle opere lapidee (tombali) derivate dai resti di edifici cittadini ed eseguite da maestranze venete, lombarde e toscane; questi reperti testimoniano la vivacità culturale della famiglia Sforza.
		La figlia di un appartenente alla famiglia Sforza, di nome Battista, sposò inoltre il conte Federico da Montefeltro di Urbino l'8 febbraio del 1460 a Pesaro. A testimonianza di ciò, alle nostre spalle (alla fine della sala dedicata alla collezione Toschi Mosca) possiamo notare il ritratto dei profili dei due eseguito tramite bassorilievo scultoreo. Il matrimonio fu felice, nonostante lui fosse il figlio della matrigna di lei; le spiccate doti culturali di entrambi li fecero vivere andando molto d'accordo.
		Mentre il marito era assente, lei infatti aveva la funzione di suo vicario di corte, carica di grande importanza per l'epoca.
		
		\section{Opere pittoriche Sala Luci e Sguardi} 
		
		\subsection{Adorazione dei Pastori di Raffaellino del Colle}
		In quest'opera possiamo osservare Maria che solleva il velo e lascia in vista il corpo del Bambino, esposto all'adorazione da parte di S. Giuseppe e dei pastori, mentre sullo sfondo collinare un angelo annuncia con gioia il lieto evento.\\
		La linea, come intagliata nel legno, dona un effetto scultoreo ai volti ed ai panneggi; lo stile cangiante e la tipizzazione umana, resi dalla pittura ad olio, donano all'opera un'aura a metà tra il quotidiano e il fiabesco.\\
		Raffaellino del Colle era un pittore originario di Borgo S. Sepolcro; entrò nella cerchia di Raffaello mentre si trovava a Roma e collaborò con Giulio Romano.
		La sua attività si sviluppò nell'ambiente dell'entroterra marchigiano, umbro e toscano e contribuì alla diffusione del Raffaellismo.
		Questo dipinto proviene dalla chiesa di S. Michele Arcangelo a Lamoli; prima di far parte della collezione Mosca era appartenuto alla nobile famiglia Passionei di Fossombrone.\\
		L'artista lavorò anche a Pesaro presso Maria I della Rovere, con il quale partecipò al cantiere degli affreschi di Villa Imperiale sul colle S. Bartolo, magnifico esempio rinascimentale di architettura suburbana.
		
		\begin{minipage}{\linewidth}
			\centering
			\vspace*{0.2cm}
			\zoombox{\includegraphics[scale=0.1]{Raffaellino_Del_Colle-Adorazione_dei_pastori.jpg}}
			\captionof{figure}{\centering{Raffaellino Del Colle - Adorazione dei pastori}}
		\end{minipage}
		
		\subsection{Adorazione del Bambino di Domenico Beccafumi (originariamente parte della collezione Ercolani-Rossini)}
		Questo dipinto ci mostra S. Giuseppe che solleva il Bambino da terra per offrirlo all'adorazione della Vergine; alle spalle dei due sono presenti delle rovine classiche che rappresentano i tempi antichi, superati con la nascita di Cristo.
		Questa composizione insolita fa sì che l'epoca pagana sia messa a confronto con quella cristiana.
		Possiamo osservare sullo sfondo comparire inoltre, a conferma di questo intento da parte del pittore, un piccolo S. Giovanni Battista, che fu precursore di Cristo, mentre vaga in un deserto roccioso.\\
		La particolarità insolita di questo dipinto è anche quella di mostrarci un S. Giuseppe più simile a un padre e quindi più proiettato nella dimensione dell'umano piuttosto che in quella del Divino, nell'atto di sorreggere ed aiutare il suo bambino a compiere i suoi primi passi, tra l'incespicare e il raggomitolarsi, esattamente come qualunque altro bambino.
		L'inquieto Domenico Beccafumi fu definito pittore eccentrico dalla critica dell'epoca; era attivo a Siena a metà del Cinquecento. I suoi dipinti sono caratterizzati da una luce e dei colori morbidi e seducenti, armonia, grazia e una composizione dall'aria irreale e fiabesca, tutti canoni da lui appresi dal celebre Raffaello.
		
		\begin{minipage}{\linewidth}
			\centering
			\vspace*{0.2cm}
			\zoombox{\includegraphics[scale=0.1]{Beccafumi_Domenico-Adorazione_di_Gesù_Bambino.jpg}}
			\captionof{figure}{\centering{Beccafumi Domenico - Adorazione di Gesù Bambino}}
		\end{minipage}
		
		\subsection{Maddalena Osuna Giron di Federico Barocci}
		La qui ritratta bella fanciulla, ornata al collo con una gorgiera di pizzo, era una dama spagnola, duchessa di Ossuna, di cui si era innamorato nientemeno che Francesco Maria II della Rovere. A testimonianza di ciò, come si può leggere sul retro del piccolo dipinto, l'innamoramento avvenne mentre egli si tratteneva alla corte di Spagna, segno che aveva intenzione di prenderla in moglie; questo fatto è antecedente al suo matrimonio con Lucrezia d'Este.
		La bellezza della fanciulla viene ulteriormente sottolineata dal pittore nel rappresentarla con guance paffute e rosee; la liquidità degli occhi grigi e la bocca di color vermiglio le conferiscono inoltre fascino e vaghezza.
		Il dipinto è attribuibile alla scuola baroccesca, il cui capostipite era Federico Barocci, amico della corte roveresca, il cui stile divenne modello artistico e culturale dell'epoca.
		
		\begin{minipage}{\linewidth}
			\centering
			\vspace*{0.2cm}
			\zoombox{\includegraphics[scale=0.25]{Barocci_Federico-Ritratto_di_Maddalena_Osuna_Giron.jpg}}
			\captionof{figure}{\centering{Barocci Federico - Ritratto di Maddalena Osuna Giron}}
		\end{minipage}
		
		\section{Sala del Mito e Devozione}
		\subsection{La Caduta dei Giganti di Guido Reni}
		Questo dipinto è uno tra i più iconici presenti nel museo e rappresenta Giove (o Zeus) nell'atto di cacciare i giganti dall'Olimpo...
		
		\begin{minipage}{\linewidth}
			\centering
			\vspace*{0.2cm}
			\zoombox{\includegraphics[scale=0.1]{Reni_Guido-Caduta_dei_giganti.jpg}}
			\captionof{figure}{\centering{Reni Guido - Caduta dei giganti}}
		\end{minipage}
		
		\vspace*{\fill}
		% Print license shield
		\doclicenseThis
	\end{flushleft}
\end{document}