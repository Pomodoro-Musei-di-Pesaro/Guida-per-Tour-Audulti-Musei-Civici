\documentclass[12pt,a4paper]{article}
\usepackage[italian]{babel}
\usepackage[utf8]{inputenc}
\usepackage{fourier} 

% Stop hyphenation
\usepackage[none]{hyphenat}

% License
\usepackage[
type={CC},
modifier={by-nc-sa},
version={4.0},
]{doclicense}

\begin{document}
	
	\title{\textbf{\centering{Manuale per operatori}\\Tour guidato per adulti ai Musei Civici}}
	\author{Alice Balestieri}
	\date{}
	
	\maketitle
	\newpage
	
	\tableofcontents
	\newpage
	
	\section{Scalone d'ingresso}
	
	\subsection{La medusa di Ferruccio Mengaroni}
	L'imponente tondo in rilievo realizzato in ceramica è ispirato alla mitologica figura greca della Medusa raffigurata nell'attimo in cui viene decapitata da Perseo,ma è anche l’autoritratto dell'artista.\\
	L'opera fin dagli inizi della sua realizzazione si dice essere stata avvolta da una maledizione; infatti mentre l'artista si stava ritraendo, lo specchio dal quale si stava osservando si ruppe. Non pensate anche voi visitatori sia ironico se si pensa che Perseo utilizzò lo scudo proprio a mo'di specchio per sconfiggere la Medusa che sia stato peoprio uno specchio rotto a gettare sciagura sul ceramista che realizzò la Medusa.\\
	Il 13 maggio 1925 l'opera fu portata per la seconda Biennale d'Arte a Monza ma durante il trasporto nella lunga e ripida scalinata l'opera si sbilanciò, Mengaroni colto dal terrore di perdere il suo capolavoro corse incontro ad esso per sorreggerlo, ma ne venne schiacciato e morì.\\
	Mengaroni è una delle più importanti figure nell'ambito della ceramica tra la seconda metà del 800 e gli inizi del 900, il suo stile è caratterizzato dal "revival" nei confronti delle opere rinascimentali tipico dell'epoca. La sua formazione avvenne attraverso lo studio delle collezioni ceramiche presenti nei Musei Civici e la loro riproduzione per poi creare delle opere proprie attraverso lo stile dell'istoriato e delle grottesche.
	
	\section{Sala Giovanni Bellini}
	In questa sala sono presenti principalmente dipinti derivanti dall'acquisizione delle opere all'interno delle Chiese del pesarese che furono oggetto di soppressione delle congregazioni religiose che fu promossa dal Regno d'Italia nell'anno 1861.\\
	Le opere appartengono al periodo che va dal  300 al 400, ovvero tra il periodo gotico- medievale e il rinascimento. sono state realizzate in scuole attive in ambito veneto, fioretino e locale; alle quali i Malatesti (Urbino) e gli Sforza (Milano) commissionavano opere per arricchire la loro collezione in ambito artistico.\\
	Altre opere qui esposte fanno parte della collezione Ercolani-Rossini.
	
	\subsection{Incoronazione della Vergine di Giovanni Bellini}
	La celebre Pala d'altare fu realizzata a Venezia nell'anno 1475 e poi successivamente trasportata in barca (smontando e riassemblando l'opera in loco) per la Chiesa di S. Francesco meglio nota nell'ambito del pesarese come Madonna delle Grazie; fu probabilmente commissionata dagli Sforza ma non è certo poiché non sono stati ritrovati i documenti originali.\\
	È senza dubbio uno dei maggiori capolavori appartenenti al periodo del rinascimento in Italia.\\
	Una delle particolarità di quest'opera è che fu ideata come le scatole cinesi, cioè come un’opera nell'opera. Infatti, la si può leggere sia orizzontalmente che dall'alto verso il basso. La cornice contribuisce a rendere appieno la storia dell'Incoronazione di Maria da parte di Gesù Cristo, attorniata da 4 Santi, rispettivamente: S. Girolamo, S. Marco, S. Giorgio e S. Terenzio (Patrono di Pesaro). \\
	Nella parte superiore della Pala, detta Cimasa, Napoleone trafugò parte del dipinto, per poterne conservare il ricordo. Successivamente grazie all'intervento dello scultore Antonio Canova furono restituite all'Italia varie di queste opere trafugate, le quali ora sono conservate nei Musei Vaticani.\\
	L'opera fu oggetto di vari restauri tra i quali quello del 2008 in preparazione per l'esposizione in una mostra all'interno delle Scuderie del Quirinale. Per la speciale occasione venne riunita alla Cimasa la parte dove veniva raffigurata l’Imbalsamazione di Cristo.\\
	Lo stile di Giovanni Bellini (artista la cui formazione avvenne nella bottega veneta del padre)  è caratterizzato dal rendere i personaggi sacri più realistici e umani che non divini, attraverso la tecnica ad olio sviluppata dai Fiamminghi nelle Fiandre del Belgio.
	
	\subsection{Beata Michelina e Santi di Jacobello Del Fiore}
	Una antica leggenda narra che la città di Pesaro è protetta dalla profezia della Beata Michelina che recita: "Proteggerò la mia città. Essa tremerà ma non cadrà”.\\
	Beata Michelina nacque a Pesaro nel 1300 da una ricca e nobile famiglia, originaria di Farneto, ovvero l’antica famiglia "Deutaleve" che nel XV secolo aveva preso il nome di Metelli. Dopo aver ricevuto una formazione adeguata al suo stato e aver sposato un nobile (da alcuni sostenuto appartenente alla famiglia dei Malatesta) rimase vedova a venti anni e benché ricca e avvenente non volle più risposarsi.\\
	Michelina benché già molto religiosa , durante l'incontro con la  pellegrina Soriana (o Sira), avuta ospite presso la sua famiglia, venne colpita molto dalla sua bontà d'animo, dalla religiosità e distacco dalla vita Mondana. Al punto d'indurla, dopo la morte del figlio, ad abbandonare le sue ricchezze e ad intraprendere il cammino della fede con rinnovato ardore. Decise quindi di diventare “terziaria” francescana, dispensando i suoi averi in favore dei poveri, edificato Chiese , riscattando la libertà di carcerati pagandone i debiti, creando doti per zitelle ed orfani.\\
	Si ridusse così in povertà e visse di elemosine e penitenza (autoflagellandosi durante le processioni). nell'anno 1347 assieme a Beato Cecco fondò la Confraternita dell'Annunziata, alla quale donò la sua casa con lo scopo di assistere e seppellire gli infermi. In vita con le sue preghiere si narra che riuscì a far evitare catastrofi e morti.\\
	Nel 19 giugno dell'anno 1356 consumata dai digiuni e dalle sofferenze, si ammalò gravemente e dopo aver giurato di proteggere Pesaro, morì. La salma rinvenuta presso un vicolo di fronte alla Chiesa di S. Cassiano (oggi denominata Via Michelina Metelli) venne trasferita presso la Chiesa di S. Francesco (ora Madonna delle Grazie).\\
	Questo Polittico ligneo in cui la santa è ritratta al centro, emergendo con forza plastica dalla scultura lignea, attorniata da archetti a sesto acuto dal gusto tardo gotico (prospero nell'area adriatica agli inizi del 400). Venne realizzata dal veneziano Jacobello Del Fiore realizzò per la Cappella della Michelina, situata nella Chiesa di S. Francesco, fortemente voluta da Pandolfo Secondo Malatesta che fece realizzare anche un'urna in marmo nell'anno 1708 per la santa poiché salvatosi da un naufragio ricorrendo all'aiuto in preghiera della santa, sciogliendo così il voto che aveva con Lei.\\
	Si narra che la stessa Santa scampò miracolosamente a un naufragio in Palestina.
	
	\subsection{S. Terenzio di Antonio Bellinzoni da Pesaro}
	Il dipinto costituiva il coperchio del Sarcofago di S. Terenzio come testimonia l’antico manoscritto incollato nella parte sinistra dello stesso.\\
	San Terenzio è rappresentato (in linea con l'iconografia del 400) come un nobile giovane che regge in una mano un libro e nell'altra la palma del martirio. In alternativa poteva essere raffigurato nelle vesti di un soldato romano, come nella Pala di Giovanni Bellini. Diversamente l’iconografia medievale lo rappresentava come vescovo.\\
	In quest'opera l'eleganza della linea rivela ancora un influsso tardo gotico ma che si sta aprendo alla novità della ricerca volumetrica e della geometria dell'area Fiorentina.\\
	L’ artista pesarese Antonio Bellinzoni (trasferitosi a Pesaro nell'anno 1410, pare da Parma) ebbe la sua formazione in Emilia presso gli ultimi maestri del tardo gotico. Influenzato dallo stile naturalistico padano usato nella bottega del padre, operò prevalentemente nelle marche dove vennero rinvenute svariate delle sue opere.
	
	\subsection{Sogno della Vergine di Michele di Matteo}
	L'opera, Realizzata da Michele di Matteo nell'anno 1410 (pittore di origini probabilmente bolognesi che operò soprattutto in Italia), appartenente alla Collezione Ercolani Rossini; raffigura, su di un pregiato fondale decorato con foglie d'oro, la Vergine addormentata intenta a sognare la tragica "Cacciata dal Paradiso di Adamo ed Eva".\\
	Secondo Alberto Longhi (celebre critico d’ arte) l'opera è caratterizzata da uno stile delicato e allo stesso tempo pungente, visibile sopratutto dalle aureole punzonate poste sui capi dei personaggi.\\
	L'opera ebbe un discreto successo nel 300, infatti venne notata dal celebre Rossini, il quale, amava le decorazioni in oro, come anche la rappresentazione di gioielli all'interno dell'opera e anche il colore rosso vermiglio presente nei panneggi (il colore era ottenuto dalla particolare lavorazione dei vermi, per questo motivo si chiama "vermiglio").  
	
	\section{Sala delle ceramiche della collezione di Domenico Mazza}
	Questa sala ospita opere di varia natura e di ampia fascia temporale, provenienti da collezioni private, di cui la più importante è quella di Domenico Mazza (originario di Urbino).\\ 
	Le Ceramiche sono decorate con la raffinata tecnica della Maiolica, tipico dello stile rinascimentale del 400 nel territorio del pesarese, di Urbania (chiamato pure Castedurante) e di Urbino; si possono notare le Ceramiche con decoro floreale (Rosa) tipico della maiolica pesarese, il cui stile ebbe un forte "revival" nel 700 con il recupero delle arti applicate.\\
	Tutt'ora l'arte della ceramica è una delle eccellenze artistiche tipiche del pesarese.
	
	\subsection{Caccia al Cinghiale Caledonio della bottega di Lanfranco dalle Gabicce}
	Questa coppa in ceramica ritrae il mitologico episodio di Meleagro intento nella caccia del cinghiale Caledonio.\\
	Da notare il dinamismo dalla pennellata veloce ed energica a confronto con impervie ed enormi montagne ed improbabili edifici.\\
	Questa coppa è caratterizzata da colori accesi, una linea "nervosa" e dinamica che rende perfettamente la velocità della scena rappresentata e ci lascia assaporare il suo gusto tipicamente manierista.\\
	Nel 1451 Guidubaldo secondo della Rovere, diede nuova importanza alla lavorazione della ceramica e fece rinascere questa arte nel 500 a Urbino.
	
	\subsection{Ercole, Anteo e Caco steso a terra della bottega di Fontana da Urbino}
	Il piatto rappresenta Ercole, muscoloso e possente, i cui nemici sono ormai esanimi.\\ 
	Questo tipo di illustrazione è stato ispirato dalle opere di Raffaello presenti nei Musei Vaticani (Domus Aurea Nerone), per questo motivo lo stile prende il nome di Raffaellesco-Grottesco. Nel XVI secolo lo stile Raffaellesco prese il sopravvento sullo stile Istoriato tipico del 500 a Urbino.
	
	\subsection{Faustina Bella, Coppa con profilo di donna della bottega di Casteldurante (Urbania)}
	Il volto marmoreo cinto da un elmo, il busto attorniato da una ricca ghirlanda e i colori che permettono al soggetto di emergere dallo sfondo blu, fanno parte della serie Belle Donne, che in questo caso particolare ritrae Faustina (opera del 1522).\\
	È stato il primo esemplare datato di cui erano parte i cosiddetti “Doni Amatori” (ovvero doni di fidanzamento o nozze per la signora), nei quali era solito rappresentare donne i cui volti avevano poche varianti, con l'aggiunta dell'aggettivo "bella" o altri complimenti riguardanti le doti della consorte. Questi erano doni di corteggiamento (considerati eleganti e di pregio) che i nobiluomini facevano alla donna amata o voluta per interessi di famiglia.
	
	\subsection{San Giuda Taddeo di Nicola da Urbino}
	L'iconografia deriva da un’incisione di Marcantonio Raimondi ispirata a un disegno di Raffaello.\\
	l personaggio rappresentato è San Giuda Taddeo, la cui identificazione è definita dall'alabarda(lo strumento del suo martirio).\\
	Lo stile di Nicola da Urbino è caratterizzato da finezza, grazia, ariosità dei panneggi e dall'intensità contrapposta alla dolcezza dello sguardo.\\
	Il suo operato è stato attivo dal 1520 al 1538.\\
	La firma sul retro è dell'illustratore Mastro Giorgio Andreoli da Gubbio, la cui cittadina faceva parte dei possedimenti del ducato di Urbino che era uno dei grandi centri di produzione delle ceramiche.
	
	\subsection{San Girolamo Penitente di Mastro Giorgio da Gubbio (1522)}
	Il Vassoio umbonato (che ricorda come forma la borchia degli scudi oltre al fatto che è stata usata come riferimento per nominare il capo di alcuni tipi di funghi) articola la scena in due spazi; l’iconografia del Santo in adorazione del Crocifisso è collocata nello spazio più esterno del vassoio, mentre al centro è rappresentato l’attributo del Santo cioè il Leone. \\
	La particolare lavorazione di questa ceramica è stata eseguita con l’antica tecnica del Lustro attribuibile al IX- X secolo d. C., che attraverso l’applicazione di speciali impasti d'ossido d'argento e rame, e una complessa tecnica di cottura rende possibile ottenere sfumature particolarmente luminose di color oro o rosso rubino.\\
	Questa complessa lavorazione venne ripresa nel XV secolo a Gubbio (Gheruta) grazie all'incentivo del ducato urbinate.
	
	\subsection{Adorazione dei Pastori di Francesco Xanto Avelli}
	Francesco Xanto Avelli rappresenta attraverso questa maiolica rinascimentale (che trae spunto da un disegno del celebre Parmigianino.) un gruppo di pastori che converge verso la vergine seduta, ovvero verso il centro focale della rappresentazione.\\
	La maiolica rinascimentale si differenzia dagli altri generi per l'uso di colori come: il blu, il verde, l’arancione, il marrone. I quali sono usati in gran misura nella composizione.\\
	Si può notare che al basamento della colonna è riportata la data di realizzazione: 1537 e sul gradino sottostante è presente la sigla dell'autore.\\
	Francesco Xanto Avelli è originario di Rovigo (anni 20 del 500), fu allievo di Nicola da Urbino che lo descrive come un personaggio eclettico e singolare. Si dice che compose persino un’opera in versi dedicata al duca Francesco Maria 1° della Rovere.\\
	Frequentemente traeva spunto da stampe dei grandi maestri del Rinascimento: Raffaello, Taddeo Zuccari e Battista Franco. Inserendo le suddette (com'era tipico dello stile Istoriato Roveresco) all'interno di paesaggi d'invenzione dell'artista.
	
	\section{Sala arredi e sculture della collezione Mosca}
	In questa sala è presente la collezione di oggetti ed arredi del periodo tra 800 e 900 appartenenti alla marchesa  Toschi Mosca di cui possiamo ammirare il ritratto appena si entra all'interno del Museo sulla destra oltrepassando il punto di accettazione. La marchesa era un'amante del gusto antiquario e qui esposta in maniera permanente come da lei voluto possiamo ammirare la sua collezione di oggetti di uso domestico personale, l'intento della marchesa era infatti quello di creare un Museo d'arte industriale accessibile al pubblico a Pesaro, dove ella stessa viveva, per far sì che altri giovani artisti si potessero ispirare alle bellezze del passato per creare i loro capolavori.
	\subsection{Stipi con vedute di Roma di Jhoann Willhelm Baur}
	In questa sala dedicata agli arredi della marchesa possiamo ammirare i pregiati Stipi con Vedute di Roma e ne possiamo notare l'usanza di decorare queste cassettiere di uso nobiliare con delle placche raffiguranti vedute dell'urbe (città), questa usanza viene fatta risalire agli anni sessanta del Seicento e serviva a far sfoggio da parte dei nobili con i loro ospiti dei loro costosi viaggi. In questi stipi in particolare sono ritratte le stupende vedute di Roma dipinte con l'utilizzo delle tempere su pergamena in maniera estremamente realistica e con minuzia di particolari dallo strasburghese Jhoann Willhelm Baur, pittore del periodo barocco presente sul luogo (Roma) che collaborò con Giacomo Herman ebanista tra il 1665 e il 1667 per la realizzazione degli stipi, possiamo notare osservandoli con attenzione la firma dell'ebanista all'interno del mobile.
	\subsection{Orologio Notturno di Andrea Nattan}
	L'orologio Notturno è un'oggetto di arredo domestico-funzionale che la marchesa custodiva nella propria camera da letto e le permetteva di leggere attraverso la luce di una candela che illuminava un forellino l'ora sorretta da un'illustrazione di un putto nel mobile (la candela veniva periodicamente sostituita dalla servitù), la marchesa si dilettava così guardando il bel mobile la sera poichè soffriva d'insonnia ed esso la aiutava a comprendere quando fosse giunto il momento di destarsi dal letto o meno. L'oggetto d'arredo era una grande cassa in legno pregiato di epoca Barocca (diciannovesimo secolo).La scritta sorretta in primo piano dall'angioletto recita:¨Volat irreparabile Tempus¨e significa: ¨Il tempo vola inesorabilmente¨ ed era attorniato dall'allegoria delle quattro stagioni, sovrastate dalla figura di un'uomo barbuto alato che rappresentava appunto il tempo.
	Fanno parte inoltre della collezione della marchesa anche oggetti in madreperla,avorio, uno specchio di pregevole fattura in vetro soffiato di Murano (Venezia) con decorazioni incise in argento di uva, varie cornici di epoca barocca ed infine varie sculture.
	\section{Sforza signori di Pesaro nella seconda metà del 400}
	Lo stemma con leone rampante e ramo di cotogno (mela ravennate,ramo di famiglia) tipico della casata è presente sulle opere lapidee (tombali)
	
\end{document}