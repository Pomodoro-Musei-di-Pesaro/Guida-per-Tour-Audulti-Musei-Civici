\documentclass[12pt,a4paper]{article}
\usepackage[italian]{babel}
\usepackage[utf8]{inputenc}
\usepackage{fourier} 

% Stop hyphenation
\usepackage[none]{hyphenat}

\begin{document}
	
	\title{\textbf{\centering{Manuale per operatori.}\\Tour guidato per adulti ai Musei Civici.}}
	\author{Alice Balestieri}
	\date{}
	
	\maketitle
	\newpage
	
	\tableofcontents
	\newpage
	
	\section{La medusa di Ferruccio Mengaroni.}
	L'imponente opera in ceramica è ispirata alla mitologica figura greca della Medusa ma è anche l’autoritratto dell'artista.\\
	L'opera fin dagli inizi della sua realizzazione si dice essere stata avvolta da una maledizione; infatti, mentre l'artista si stava ritraendo, lo specchio dal quale si stava osservando si ruppe.\\
	Il 13 maggio 1925 l'opera fu portata per la seconda Biennale d'Arte a Monza ma durante il trasporto nella lunga e ripida scalinata l'opera si sbilanciò, Mengaroni colto dal terrore di perdere il suo capolavoro corse incontro ad essa per sorreggerla, ne venne schiacciato e morì.\\
	\underline{Mengaroni è una delle più importanti figure nell'ambito della ceramica} tra la seconda metà dell’800 e gli inizi del 900, il suo stile è caratterizzato \underline{dal "revival" nei confronti delle opere rinascimentali} tipico dell'epoca. La sua formazione avvenne attraverso lo studio delle collezioni ceramiche presenti nei Musei Civici e \underline{la loro riproduzione per poi creare delle opere proprie} attraverso lo stile \underline{dell'istoriato e delle grottesche.}
	
	\section{Sala Giovanni Bellini.}
	In questa sala sono presenti principalmente dipinti \underline{derivanti dall'acquisizione delle opere all'interno delle Chiese del pesarese che furono oggetto di soppressione delle congregazioni religiose che fu promossa dal Regno d'Italia nell'anno 1861.}\\
	Le opere appartengono al periodo che va dagli anni  300 ai 400, ovvero tra il periodo gotico- medievale e il rinascimento. sono state realizzate in \underline{scuole attive in ambito veneto, fioretino e locale}; alle quali i Malatesti (Urbino) e gli Sforza (Milano) commissionavano opere per arricchire la loro collezione in ambito artistico.\\
	Altre opere qui esposte fanno parte della collezione \textbf{Ercolani-Rossini.}
	
	\section{Incoronazione della Vergine di Giovanni Bellini.}
	La celebre \underline{Pala d'altare} fu realizzata a Venezia nell'anno 1475 e poi successivamente trasportata in barca (smontando e riassemblando l'opera in loco) per la Chiesa di S. Francesco meglio nota nell'ambito del pesarese come Madonna delle Grazie; fu probabilmente commissionata dagli Sforza ma non è certo poiché non sono stati ritrovati i documenti originali.\\
	È senza dubbio uno dei maggiori capolavori appartenenti al periodo del rinascimento in Italia.\\
	Una delle particolarità di quest'opera è che fu ideata come le scatole cinesi, \underline{cioè come un’opera nell'opera.} Infatti, la si può leggere sia orizzontalmente che dall'alto verso il basso. \underline{La cornice contribuisce a rendere appieno la storia} dell'Incoronazione di Maria da parte di Gesù Cristo, attorniata da 4 Santi, rispettivamente: S. Girolamo, S. Marco, S. Giorgio e S. Terenzio (Patrono di Pesaro). \\
	Nella parte superiore della Pala, detta Cimasa, Napoleone trafugò parte del dipinto, per poterne conservare il ricordo. Successivamente grazie all'intervento dello scultore Antonio Canova furono restituite all'Italia varie di queste opere trafugate, le quali ora sono conservate nei Musei Vaticani.\\
	L'opera fu oggetto di vari restauri tra i quali quello del 2008 in preparazione per l'esposizione in una mostra all'interno delle Scuderie del Quirinale. Per la speciale occasione venne riunita alla Cimasa la parte dove veniva raffigurata l’Imbalsamazione di Cristo.\\
	Lo stile di Giovanni Bellini (\underline{artista la cui formazione avvenne nella bottega veneta del padre})  è caratterizzato dal rendere i personaggi sacri più realistici e umani che non divini, attraverso la tecnica ad olio sviluppata dai Fiamminghi nelle Fiandre del Belgio.
	
	\section{Beata Michelina e Santi di Jacobello Del Fiore.}
	Una antica leggenda narra che la città di Pesaro è protetta dalla profezia della Beata Michelina che recita: "Proteggerò la mia città. Essa tremerà ma non cadrà”.\\
	\underline{Beata Michelina nacque a Pesaro nel 1300} da una ricca e nobile famiglia, originaria di Farneto, ovvero l’antica famiglia "Deutaleve" che nel XV secolo aveva preso il nome di Metelli. Dopo aver ricevuto una \underline{formazione adeguata al suo stato} e aver sposato un nobile (da alcuni sostenuto appartenente alla famiglia dei Malatesta) rimase vedova a 20 anni e benché ricca e avvenente non volle più risposarsi.\\
	Michelina benché già molto religiosa , durante l'incontro con la  pellegrina Soriana (o Sira), avuta ospite presso la sua famiglia, venne colpita molto dalla sua bontà d'animo , religiosità e distacco dalla vita Mondana. Al punto d'indurla, dopo la morte del figlio, ad abbandonare le sue ricchezze e ad intraprendere il cammino della fede con rinnovato ardore. Decise quindi di diventare “terziaria” francescana, dispensando i suoi averi in favore dei poveri, edificato Chiese , riscattando la libertà di carcerati pagandone i debiti, creando doti per zitelle ed orfani.\\
	Si ridusse così in povertà e visse di elemosine e penitenza (autoflagellandosi durante le processioni). nell'anno 1347 assieme a Beato Cecco fondò la Confraternita dell’Annunziata, alla quale donò la sua casa con lo scopo di assistere e seppellire gli infermi. In vita con le sue preghiere si narra che riuscì a far evitare catastrofi e morti.\\
	Nel 19 giugno dell'anno 1356 consumata dai digiuni e dalle sofferenze, si ammalò gravemente e dopo aver giurato di proteggere Pesaro, morì. La salma rinvenuta presso un vicolo di fronte alla Chiesa di S. Cassiano (oggi denominata Via Michelina Metelli) venne trasferita presso la Chiesa di S. Francesco (ora Madonna delle Grazie).\\
	Questo Polittico ligneo in cui la santa è ritratta al centro, emergendo con forza plastica dalla scultura lignea, attorniata da archetti a sesto acuto dal gusto tardo gotico (prospero nell’ area adriatica agli inizi del 400). Venne realizzata dal veneziano Jacobello Del Fiore realizzò per la Cappella della Michelina, situata nella Chiesa di S. Francesco, fortemente voluta da Pandolfo Secondo Malatesta che fece realizzare anche un'urna in marmo nell'anno 1708 per la santa poiché salvatosi da un naufragio ricorrendo all'aiuto in preghiera della santa, sciogliendo così il voto che aveva con Lei.\\
	Si narra che la stessa Santa scampò miracolosamente a un naufragio in Palestina.
	
\end{document}