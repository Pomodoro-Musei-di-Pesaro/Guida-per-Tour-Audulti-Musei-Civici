\documentclass[12pt,a4paper]{article}
\usepackage[italian]{babel}
\usepackage[utf8]{inputenc}
\usepackage{fourier} 

% Stop hyphenation
\usepackage[none]{hyphenat}

\begin{document}
	
	\title{\textbf{\centering{Manuale per operatori.}\\Tour guidato per adulti ai Musei Civici.}}
	\author{Alice Balestieri}
	\date{}
	
	\maketitle
	\newpage
	
	\tableofcontents
	\newpage
	
	\section{La medusa di Ferruccio Mengaroni.}
	L'imponente opera in ceramica è ispirata alla mitologica figura greca della Medusa ma è anche l’autoritratto dell'artista.\\
	L'opera fin dagli inizi della sua realizzazione si dice essere stata avvolta da una maledizione; infatti, mentre l'artista si stava ritraendo, lo specchio dal quale si stava osservando si ruppe.\\
	Il 13 maggio 1925 l'opera fu portata per la seconda Biennale d'Arte a Monza ma durante il trasporto nella lunga e ripida scalinata l'opera si sbilanciò, Mengaroni colto dal terrore di perdere il suo capolavoro corse incontro ad essa per sorreggerla, ne venne schiacciato e morì.\\
	\underline{Mengaroni è una delle più importanti figure nell'ambito della ceramica} tra la seconda metà dell’800 e gli inizi del 900, il suo stile è caratterizzato \underline{dal "revival" nei confronti delle opere rinascimentali} tipico dell'epoca. La sua formazione avvenne attraverso lo studio delle collezioni ceramiche presenti nei Musei Civici e \underline{la loro riproduzione per poi creare delle opere proprie} attraverso lo stile \underline{dell'istoriato e delle grottesche.}
	
	\section{Sala Giovanni Bellini.}
	In questa sala sono presenti principalmente dipinti \underline{derivanti dall'acquisizione delle opere all'interno delle Chiese del pesarese che furono oggetto di soppressione delle congregazioni religiose che fu promossa dal Regno d'Italia nell'anno 1861.}\\
	Le opere appartengono al periodo che va dagli anni  300 ai 400, ovvero tra il periodo gotico- medievale e il rinascimento. sono state realizzate in \underline{scuole attive in ambito veneto, fioretino e locale}; alle quali i Malatesti (Urbino) e gli Sforza (Milano) commissionavano opere per arricchire la loro collezione in ambito artistico.\\
	Altre opere qui esposte fanno parte della collezione \textbf{Ercolani-Rossini.}
	
	\section{Incoronazione della Vergine di Giovanni Bellini.}
	La celebre \underline{Pala d'altare} fu realizzata a Venezia nell'anno 1475 e poi successivamente trasportata in barca (smontando e riassemblando l'opera in loco) per la Chiesa di S. Francesco meglio nota nell'ambito del pesarese come Madonna delle Grazie; fu probabilmente commissionata dagli Sforza ma non è certo poiché non sono stati ritrovati i documenti originali.\\
	
\end{document}