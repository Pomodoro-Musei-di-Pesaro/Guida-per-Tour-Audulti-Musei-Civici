\documentclass[hidelinks,12pt,a4paper]{article}
\usepackage[italian]{babel}
\usepackage[utf8]{inputenc}
\usepackage{fourier} 

% To avoid GitHub Action error
\usepackage{hyperref}

% Stop hyphenation
\usepackage[none]{hyphenat}

% Justifying text
\emergencystretch 3em

% Enlarge section & subsection
\usepackage{titlesec}
\titleformat*{\section}{\LARGE\bfseries}
\titleformat*{\subsection}{\Large\bfseries}

% Remove first empty page
\usepackage{atbegshi}
\AtBeginDocument{\AtBeginShipoutNext{\AtBeginShipoutDiscard}}

% License
\usepackage[
type={CC},
modifier={by-nc-sa},
version={4.0},
]{doclicense}

\begin{document}
	\begin{flushleft}
	
		%Enlarge text
		\LARGE
	
		\title{\textbf{\centering{Manuale per operatori}\\Tour guidato per adulti ai Musei Civici}}
		\author{Alice Balestieri\\Francesco Rombaldoni}
		\date{}
		
		\maketitle
		
		% Adjust page counter
		\setcounter{page}{1}
		\newpage
		\topskip0pt
		\vspace*{\fill}
		Aggiungere qui la descrizione riguardo il documento. 
		\vspace*{\fill}
		\newpage
		\tableofcontents
		\newpage
		
		\section{Scalone d'ingresso}
		
		\subsection{La medusa di Ferruccio Mengaroni}
		L'imponente tondo in rilievo realizzato in ceramica è ispirato alla mitologica figura greca della Medusa raffigurata nell'attimo in cui viene decapitata da Perseo, ma è anche l’autoritratto dell'artista.\\
		L'opera fin dagli inizi della sua realizzazione si dice essere stata avvolta da una maledizione; infatti mentre l'artista si stava ritraendo, lo specchio dal quale si stava osservando si ruppe. Non pensate anche voi visitatori sia ironico se si pensa che Perseo utilizzò lo scudo proprio a mo'di specchio per sconfiggere la Medusa che sia stato peoprio uno specchio rotto a gettare sciagura sul ceramista che realizzò la Medusa.\\
		Il 13 maggio 1925 l'opera fu portata per la seconda Biennale d'Arte a Monza ma durante il trasporto nella lunga e ripida scalinata l'opera si sbilanciò, Mengaroni colto dal terrore di perdere il suo capolavoro corse incontro ad esso per sorreggerlo, ma ne venne schiacciato e morì.\\
		Mengaroni è una delle più importanti figure nell'ambito della ceramica tra la seconda metà del 800 e gli inizi del 900, il suo stile è caratterizzato dal "revival" nei confronti delle opere rinascimentali tipico dell'epoca. La sua formazione avvenne attraverso lo studio delle collezioni ceramiche presenti nei Musei Civici e la loro riproduzione per poi creare delle opere proprie attraverso lo stile dell'istoriato e delle grottesche.
		
		\section{Sala Giovanni Bellini}
		In questa sala sono presenti principalmente dipinti derivanti dall'acquisizione delle opere all'interno delle Chiese del pesarese che furono oggetto di soppressione delle congregazioni religiose che fu promossa dal Regno d'Italia nell'anno 1861.\\
		Le opere appartengono al periodo che va dal  300 al 400, ovvero tra il periodo gotico- medievale e il rinascimento. Sono state realizzate in scuole attive in ambito veneto, fioretino e locale; alle quali i Malatesti (Urbino) e gli Sforza (Milano) commissionavano opere per arricchire la loro collezione in ambito artistico.\\
		Altre opere qui esposte fanno parte della collezione Ercolani-Rossini.
		
		\subsection{Incoronazione della Vergine di Giovanni Bellini}
		La celebre Pala d'altare fu realizzata a Venezia nell'anno 1475 e poi successivamente trasportata in barca (smontando e riassemblando l'opera in loco) per la Chiesa di S. Francesco meglio nota nell'ambito del pesarese come Madonna delle Grazie; fu probabilmente commissionata dagli Sforza ma non è certo poiché non sono stati ritrovati i documenti originali.\\
		È senza dubbio uno dei maggiori capolavori appartenenti al periodo del rinascimento in Italia.\\
		Una delle particolarità di quest'opera è che fu ideata come le scatole cinesi, cioè come un’opera nell'opera. Infatti, la si può leggere sia orizzontalmente che dall'alto verso il basso. La cornice contribuisce a rendere appieno la storia dell'Incoronazione di Maria da parte di Gesù Cristo, attorniata da 4 Santi, rispettivamente: S. Girolamo, S. Marco, S. Giorgio e S. Terenzio (Patrono di Pesaro). \\
		Nella parte superiore della Pala, detta Cimasa, Napoleone trafugò parte del dipinto, per poterne conservare il ricordo. Successivamente grazie all'intervento dello scultore Antonio Canova furono restituite all'Italia varie di queste opere trafugate, le quali ora sono conservate nei Musei Vaticani.\\
		L'opera fu oggetto di vari restauri tra i quali quello del 2008 in preparazione per l'esposizione in una mostra all'interno delle Scuderie del Quirinale. Per la speciale occasione venne riunita alla Cimasa la parte dove veniva raffigurata l’Imbalsamazione di Cristo.\\
		Lo stile di Giovanni Bellini (artista la cui formazione avvenne nella bottega veneta del padre)  è caratterizzato dal rendere i personaggi sacri più realistici e umani che non divini, attraverso la tecnica ad olio sviluppata dai Fiamminghi nelle Fiandre del Belgio.
		
		\subsection{Beata Michelina e Santi di Jacobello Del Fiore}
		Una antica leggenda narra che la città di Pesaro è protetta dalla profezia della Beata Michelina che recita: "Proteggerò la mia città. Essa tremerà ma non cadrà”.\\
		Beata Michelina nacque a Pesaro nel 1300 da una ricca e nobile famiglia, originaria di Farneto, ovvero l’antica famiglia "Deutaleve" che nel XV secolo aveva preso il nome di Metelli. Dopo aver ricevuto una formazione adeguata al suo stato e aver sposato un nobile (da alcuni sostenuto appartenente alla famiglia dei Malatesta) rimase vedova a venti anni e benché ricca e avvenente non volle più risposarsi.\\
		Michelina benché già molto religiosa, durante l'incontro con la  pellegrina Soriana (o Sira), avuta ospite presso la sua famiglia, venne colpita molto dalla sua bontà d'animo, dalla religiosità e distacco dalla vita Mondana. Al punto d'indurla, dopo la morte del figlio, ad abbandonare le sue ricchezze e ad intraprendere il cammino della fede con rinnovato ardore. Decise quindi di diventare “terziaria” francescana, dispensando i suoi averi in favore dei poveri, edificato Chiese, riscattando la libertà di carcerati pagandone i debiti, creando doti per zitelle ed orfani.\\
		Si ridusse così in povertà e visse di elemosine e penitenza (autoflagellandosi durante le processioni). nell'anno 1347 assieme a Beato Cecco fondò la Confraternita dell'Annunziata, alla quale donò la sua casa con lo scopo di assistere e seppellire gli infermi. In vita con le sue preghiere si narra che riuscì a far evitare catastrofi e morti.\\
		Nel 19 giugno dell'anno 1356 consumata dai digiuni e dalle sofferenze, si ammalò gravemente e dopo aver giurato di proteggere Pesaro, morì. La salma rinvenuta presso un vicolo di fronte alla Chiesa di S. Cassiano (oggi denominata Via Michelina Metelli) venne trasferita presso la Chiesa di S. Francesco (ora Madonna delle Grazie).\\
		Questo Polittico ligneo in cui la santa è ritratta al centro, emergendo con forza plastica dalla scultura lignea, attorniata da archetti a sesto acuto dal gusto tardo gotico (prospero nell'area adriatica agli inizi del 400). Venne realizzata dal veneziano Jacobello Del Fiore realizzò per la Cappella della Michelina, situata nella Chiesa di S. Francesco, fortemente voluta da Pandolfo Secondo Malatesta che fece realizzare anche un'urna in marmo nell'anno 1708 per la santa poiché salvatosi da un naufragio ricorrendo all'aiuto in preghiera della santa, sciogliendo così il voto che aveva con Lei.\\
		Si narra che la stessa Santa scampò miracolosamente a un naufragio in Palestina.
		
		\subsection{S. Terenzio di Antonio Bellinzoni da Pesaro}
		Il dipinto costituiva il coperchio del Sarcofago di S. Terenzio come testimonia l’antico manoscritto incollato nella parte sinistra dello stesso.\\
		San Terenzio è rappresentato (in linea con l'iconografia del 400) come un nobile giovane che regge in una mano un libro e nell'altra la palma del martirio. In alternativa poteva essere raffigurato nelle vesti di un soldato romano, come nella Pala di Giovanni Bellini. Diversamente l’iconografia medievale lo rappresentava come vescovo.\\
		In quest'opera l'eleganza della linea rivela ancora un influsso tardo gotico ma che si sta aprendo alla novità della ricerca volumetrica e della geometria dell'area Fiorentina.\\
		L'artista pesarese Antonio Bellinzoni (trasferitosi a Pesaro nell'anno 1410, pare da Parma) ebbe la sua formazione in Emilia presso gli ultimi maestri del tardo gotico. Influenzato dallo stile naturalistico padano usato nella bottega del padre, operò prevalentemente nelle marche dove vennero rinvenute svariate delle sue opere.
		
		\subsection{Sogno della Vergine di Michele di Matteo}
		L'opera, Realizzata da Michele di Matteo nell'anno 1410 (pittore di origini probabilmente bolognesi che operò soprattutto in Italia), appartenente alla Collezione Ercolani Rossini; raffigura, su di un pregiato fondale decorato con foglie d'oro, la Vergine addormentata intenta a sognare la tragica "Cacciata dal Paradiso di Adamo ed Eva".\\
		Secondo Alberto Longhi (celebre critico d’ arte) l'opera è caratterizzata da uno stile delicato e allo stesso tempo pungente, visibile sopratutto dalle aureole punzonate poste sui capi dei personaggi.\\
		L'opera ebbe un discreto successo nel 300, infatti venne notata dal celebre Rossini, il quale, amava le decorazioni in oro, come anche la rappresentazione di gioielli all'interno dell'opera e anche il colore rosso vermiglio presente nei panneggi (il colore era ottenuto dalla particolare lavorazione dei vermi, per questo motivo si chiama "vermiglio").  
		
		\section{Sala delle ceramiche della collezione di Domenico Mazza}
		Questa sala ospita opere di varia natura e di ampia fascia temporale, provenienti da collezioni private, di cui la più importante è quella di Domenico Mazza (originario di Urbino).\\ 
		Le Ceramiche sono decorate con la raffinata tecnica della Maiolica, tipico dello stile rinascimentale del 400 nel territorio del pesarese, di Urbania (chiamato pure Castedurante) e di Urbino; si possono notare le Ceramiche con decoro floreale (Rosa) tipico della maiolica pesarese, il cui stile ebbe un forte "revival" nel 700 con il recupero delle arti applicate.\\
		Tutt'ora l'arte della ceramica è una delle eccellenze artistiche tipiche del pesarese.
		
		\subsection{Caccia al Cinghiale Caledonio della bottega di Lanfranco dalle Gabicce}
		Questa coppa in ceramica ritrae il mitologico episodio di Meleagro intento nella caccia del cinghiale Caledonio.\\
		Da notare il dinamismo dalla pennellata veloce ed energica a confronto con impervie ed enormi montagne ed improbabili edifici.\\
		Questa coppa è caratterizzata da colori accesi, una linea "nervosa" e dinamica che rende perfettamente la velocità della scena rappresentata e ci lascia assaporare il suo gusto tipicamente manierista.\\
		Nel 1451 Guidubaldo secondo della Rovere, diede nuova importanza alla lavorazione della ceramica e fece rinascere questa arte nel 500 a Urbino.
		
		\subsection{Ercole, Anteo e Caco steso a terra della bottega di Fontana da Urbino}
		Il piatto rappresenta Ercole, muscoloso e possente, i cui nemici sono ormai esanimi.\\ 
		Questo tipo di illustrazione è stato ispirato dalle opere di Raffaello presenti nei Musei Vaticani (Domus Aurea Nerone), per questo motivo lo stile prende il nome di Raffaellesco-Grottesco. Nel XVI secolo lo stile Raffaellesco prese il sopravvento sullo stile Istoriato tipico del 500 a Urbino.
		
		\subsection{Faustina Bella, Coppa con profilo di donna della bottega di Casteldurante (Urbania)}
		Il volto marmoreo cinto da un elmo, il busto attorniato da una ricca ghirlanda e i colori che permettono al soggetto di emergere dallo sfondo blu, fanno parte della serie Belle Donne, che in questo caso particolare ritrae Faustina (opera del 1522).\\
		È stato il primo esemplare datato di cui erano parte i cosiddetti “Doni Amatori” (ovvero doni di fidanzamento o nozze per la signora), nei quali era solito rappresentare donne i cui volti avevano poche varianti, con l'aggiunta dell'aggettivo "bella" o altri complimenti riguardanti le doti della consorte. Questi erano doni di corteggiamento (considerati eleganti e di pregio) che i nobiluomini facevano alla donna amata o voluta per interessi di famiglia.
		
		\subsection{San Giuda Taddeo di Nicola da Urbino}
		L'iconografia deriva da un’incisione di Marcantonio Raimondi ispirata a un disegno di Raffaello.\\
		Il personaggio rappresentato è San Giuda Taddeo, la cui identificazione è definita dall'alabarda(lo strumento del suo martirio).\\
		Lo stile di Nicola da Urbino è caratterizzato da finezza, grazia, ariosità dei panneggi e dall'intensità contrapposta alla dolcezza dello sguardo.\\
		Il suo operato è stato attivo dal 1520 al 1538.\\
		La firma sul retro è dell'illustratore Mastro Giorgio Andreoli da Gubbio, la cui cittadina faceva parte dei possedimenti del ducato di Urbino che era uno dei grandi centri di produzione delle ceramiche.
		
		\subsection{San Girolamo Penitente di Mastro Giorgio da Gubbio (1522)}
		Il Vassoio umbonato (che ricorda come forma la borchia degli scudi oltre al fatto che è stata usata come riferimento per nominare il capo di alcuni tipi di funghi) articola la scena in due spazi; l’iconografia del Santo in adorazione del Crocifisso è collocata nello spazio più esterno del vassoio, mentre al centro è rappresentato l’attributo del Santo cioè il Leone. \\
		La particolare lavorazione di questa ceramica è stata eseguita con l’antica tecnica del Lustro attribuibile al IX- X secolo d. C., che attraverso l’applicazione di speciali impasti d'ossido d'argento e rame, e una complessa tecnica di cottura rende possibile ottenere sfumature particolarmente luminose di color oro o rosso rubino.\\
		Questa complessa lavorazione venne ripresa nel XV secolo a Gubbio (Gheruta) grazie all'incentivo del ducato urbinate.
		
		\subsection{Adorazione dei Pastori di Francesco Xanto Avelli}
		Francesco Xanto Avelli rappresenta attraverso questa maiolica rinascimentale (che trae spunto da un disegno del celebre Parmigianino) un gruppo di pastori che converge verso la vergine seduta, ovvero verso il centro focale della rappresentazione.\\
		La maiolica rinascimentale si differenzia dagli altri generi per l'uso di colori come: il blu, il verde, l’arancione, il marrone. I quali sono usati in gran misura nella composizione.\\
		Si può notare che al basamento della colonna è riportata la data di realizzazione: 1537 e sul gradino sottostante è presente la sigla dell'autore.\\
		Francesco Xanto Avelli è originario di Rovigo (anni 20 del 500), fu allievo di Nicola da Urbino che lo descrive come un personaggio eclettico e singolare. Si dice che compose persino un’opera in versi dedicata al duca Francesco Maria 1° della Rovere.\\
		Frequentemente traeva spunto da stampe dei grandi maestri del Rinascimento: Raffaello, Taddeo Zuccari e Battista Franco. Inserendo le suddette (com'era tipico dello stile Istoriato Roveresco) all'interno di paesaggi d'invenzione dell'artista.
		
		\section{Sala arredi e sculture della collezione Mosca}
		In questa sala è presente la collezione di oggetti ed arredi del periodo tra 800 e 900 appartenenti alla marchesa  Toschi Mosca di cui possiamo ammirare il ritratto appena si entra all'interno del Museo sulla destra oltrepassando il punto di accettazione. La marchesa era un'amante del gusto antiquario e qui esposta in maniera permanente come da lei voluto possiamo ammirare la sua collezione di oggetti di uso domestico personale, l'intento della marchesa era infatti quello di creare un Museo d'arte industriale accessibile al pubblico a Pesaro, dove ella stessa viveva, per far sì che altri giovani artisti si potessero ispirare alle bellezze del passato per creare i loro capolavori.
		
		\subsection{Stipi con vedute di Roma di Jhoann Willhelm Baur}
		In questa sala dedicata agli arredi della marchesa possiamo ammirare i pregiati Stipi con Vedute di Roma e ne possiamo notare l'usanza di decorare queste cassettiere di uso nobiliare con delle placche raffiguranti vedute dell'urbe (città), questa usanza viene fatta risalire agli anni sessanta del Seicento e serviva a far sfoggio da parte dei nobili con i loro ospiti dei loro costosi viaggi. In questi stipi in particolare sono ritratte le stupende vedute di Roma dipinte con l'utilizzo delle tempere su pergamena in maniera estremamente realistica e con minuzia di particolari dallo strasburghese Jhoann Willhelm Baur, pittore del periodo barocco presente sul luogo (Roma) che collaborò con Giacomo Herman ebanista tra il 1665 e il 1667 per la realizzazione degli stipi, possiamo notare osservandoli con attenzione la firma dell'ebanista all'interno del mobile.
		
		\subsection{Orologio Notturno di Andrea Nattan}
		L'orologio Notturno è un oggetto di arredo domestico-funzionale che la marchesa custodiva nella propria camera da letto e le permetteva di leggere attraverso la luce di una candela che illuminava un forellino l'ora sorretta da un'illustrazione di un putto nel mobile (la candela veniva periodicamente sostituita dalla servitù), la marchesa si dilettava così guardando il bel mobile la sera poiché soffriva d'insonnia ed esso la aiutava a comprendere quando fosse giunto il momento di destarsi dal letto o meno. L'oggetto d'arredo era una grande cassa in legno pregiato di epoca Barocca (diciannovesimo secolo). La scritta sorretta in primo piano dall'angioletto recita:"Volat irreparabile Tempus" e significa: "Il tempo vola inesorabilmente" ed era attorniato dall'allegoria delle quattro stagioni, sovrastate dalla figura di un uomo barbuto alato che rappresentava appunto il tempo.\\
		Fanno parte inoltre della collezione della marchesa anche oggetti in madreperla, avorio, uno specchio di pregevole fattura in vetro soffiato di Murano (Venezia) con decorazioni incise in argento di uva, varie cornici di epoca barocca ed infine varie sculture.
		
		\section{Sforza signori di Pesaro nella seconda metà del 400}
		Lo stemma con leone rampante e ramo di cotogno (mela ravennate, ramo di famiglia) tipico della casata è presente sulle opere lapidee (tombali) derivate dai resti di edifici cittadini ed eseguite da maestranze venete, lombarde e toscane questi reperti testimoniano la vivacità culturale della famiglia Sforza. La figlia di un appartenente alla famiglia Sforza di nome Battista sposò inoltre il conte Federico da Montefeltro di Urbino nel 8 Febbraio del 1460 a Pesaro, a testimonianza di ciò alle nostre spalle (alla fine della sala dedicata alla collezione Toschi Mosca) possiamo notare il ritratto dei profili dei due eseguito tramite bassorilievo scultoreo, il matrimonio fu felice, nonostante lui fosse il figlio della matrigna di lei; ma le spiccati doti culturali di entrambi li fecero vivere andando molto d'accordo fra loro. Mentre il marito era assente lei infatti aveva la funzione di suo vicario di corte, carica di grande importanza per l'epoca.
		
		\section{Opere pittoriche Sala Luci e Sguardi} 
		
		\subsection{Adorazione dei Pastori di Raffaellino del Colle}
		In quest'opera possiamo osservare Maria che solleva il velo e lascia in vista il corpo del bambino, esposto all'adorazione da parte di S. Giuseppe e i pastori, mentre sullo sfondo collinare un'angelo annuncia con gioia il lieto evento.\\
		La linea come intagliata nel legno dona un effetto scultoreo ai volti ed i panneggi, lo stile cangiante e la tipizzazione umana dati dalla pittura ad olio donano all'opera un'aura a metà tra il quotidiano e il fiabesco.\\
		Raffaellino del Colle era un pittore originario di Borgo S. Sepolcro, entrò nella cerchia di Raffaello mentre si trovava a Roma e collaborò con Giulio Romano. La sua attività si sviluppò nell'ambiente dell'entroterra marchigiano, umbro e toscano e contribuì alla diffusione del Raffaellismo. Questo dipinto proviene dalla chiesa S. Michele Arcangelo a Lamoli, prima di far parte della collezione Mosca era appartenuto alla nobile famiglia Passionei di Fossombrone.\\
		L'artista lavorò anche a Pesaro presso Maria Primo della Rovere con il quale partecipò al cantiere degli affreschi di Villa Imperiale sul colle S. Bartolo, magnifico esempio rinascimentale di architettura sub-urbana.
		
		\subsection{Adorazione del Bambino di Domenico Beccafumi(originariamente parte della collezione Ercolani-Rossini)}
		Questo dipinto ci mostra S. Giuseppe che solleva il bambino da terra per offrirlo all'adorazione della Vergine, alle spalle dei due sono presenti delle rovine classiche che rappresentano i tempi antichi, superati con la nascita di Cristo. Questa composizione insolita fa sì che l'epoca pagana sia messa a confronto con quella cristiana. Possiamo osservare sullo sfondo comparire inoltre a conferma di questo intento da parte del pittore, un piccolo S. Giovanni Battista che fu precursore di Cristo, mentre vaga in un deserto roccioso.\\
		La particolarità insolita di questo dipinto è anche quella di mostrarci un S. Giuseppe più simile a un padre e quindi più proiettato nella dimensione dell'umano piuttosto che non quella del Divino, nell'atto di sorreggere ed aiutare il suo bambino a compiere i suoi primi passi, tra l'incespicare e il raggomitolarsi esattamente come qualunque altro bambino. L'inquieto Domenico Beccafumi fu definito pittore eccentrico dalla critica dell'epoca, era attivo a Siena a metà del 500, i suoi dipinti sono caratterizzati da: una luce e dei colori morbidi e seducenti, armonia, grazia e una composizione dall'aria irreale e fiabesca; tutti canoni da lui appresi dal celebre Raffaello.
		
		\subsection{Maddalena Osuna Giron di Federico Barocci}
		La quì ritratta bella fanciulla ornata al collo con una gorgiera di pizzo era una dama spagnola, duchessa di Ossuno di cui si era innamorato nientemeno che Francesco Maria Secondo Feltrio della Rovere a testimonianza di ciò  come si può leggere sul retro del piccolo dipinto, l'innamoramento avvenne mentre egli si era trattenuto alla corte di Cattolica (Spagna) in segno che aveva intenzione di prenderla in moglie, questo fatto è antecedente al suo matrimonio con Lucrezia d'Este. La bellezza della fanciulla viene ulteriormente sottolineata dal pittore nel rappresentarla con guance paffute e rosee, la liquidità degli occhi grigi e la bocca di color vermiglio le conferiscono inoltre fascino e vaghezza. Il dipinto è attribuibile alla scuola barocciesca il cui capostipite era Federico Barocci amico della corte roveresca il cui stile divenne modello artistico e culturale dell'epoca.
		
		\section{Sala del Mito e Devozione}
		
		\subsection{La Caduta dei Giganti di Guido Reni}
		Questo dipinto è uno tra i più iconici presenti nel museo e rappresenta Giove o Zeus nell'atto di cacciare i giganti dall'Olimpo rispedendoli sulla Terra con un gesto risoluto del braccio, pronto a scagliare un fulmine dalla mano (questa scena è tratta dalle Metamorfosi di Ovidio).\\
		Il punto di vista ribassato dà noi l'impressione che i giganti e le pietre fuoriescano dal dipinto, questa ardita prospettiva fa pensare che l'imponente dipinto decorasse un soffitto di un Palazzo nobiliare, probabilmente il Palazzo Te di Mantova. Lo stile di Guido Reni caratterizzato solitamente da grazia, compostezza e armonia (tipico del classicismo seicentesco di Bologna che si ispirava al Raffaellismo) quì non è possibile trovarlo dato il tema bellicoso di questo quadro infatti l'artista adottò in quest'opera uno stile diametralmente opposto a quello che lo caratterizza, infatti possiamo notare quì uno stile: dinamico, concitato e carico di violenza che meglio esprime la scena.
		
		\subsection{San Giuseppe in Preghiera e Maddalena Penitente}
		Quest'opera era stata commissionata per l'Oratorio S. Filippo Neri (ora non più esistente).\\
		Procedendo sulla destra far ammirare ai visitatori la coppia di dipinti realizzati da Simone Cantarini.\\
		Nel primo S. Giuseppe posa in ginocchio con i palmi delle mani aperti e volge lo sguardo verso l'alto in segno di remissione a Dio, i personaggi sono collocati in un luogo angusto nel quale la prima luce illumina il Santo facendo pensare alla presenza del Paradiso, mentre una seconda luce che si insinua sulla sinistra ad illuminare con macabro realismo il teschio e la clessidra (che simboleggiano la Vanitas, cioè la volubilità del tempo, con il suo inesorabile scorrere che porta alla morte) questi dettagli ci riportano alla dimensione della caducità umana e quindi del reale, altro dettaglio importante per comprendere il dipinto è il bastone con il giglio che rappresenta la castità del Santo.\\
		Nel secondo dipinto Maddalena Penitente, Maddalena contempla il teschio, con accanto a sè un libro aperto per la preghiera(che indica il suo pentimento dai peccati carnali essendo lei probabilmente stata precedentemente una prostituta) ed un vaso in alabastro poiché l'aneddoto della sua conversione a Dio racconta che lei piangendo lavò i piedi a Cristo e con i suoi lunghi capelli che si cosparse di oli profumati li pulì, a causa di questo motivo è spesso rappresentata con un vaso per contenere oli essenziali.\\
		Il tutto ha come sfondo un paesaggio all'aria aperta illuminato da una fredda luce invernale che riverbera sulla pelle pallida della Santa, il libro e la chioma ramata di Maddalena contribuiscono a risaltare maggiormente il gioco di contrasti sui toni del rosso presente anche nel panneggio, mentre il fondale ha dei colori cupi che vogliono accentuare la drammaticità dell'avvenuta morte di Cristo.\\
		Giuseppe Simone Cantarini nacque proprio quì a Pesaro nel 1612 e fin dalla giovane età si notò in lui una spiccata inclinazione al disegno, egli si formò presso la bottega di Giovan Giacomo Pandolfi (anch'esso era nato ed attivo a Pesaro) e successivamente continuò ad apprendere attraverso lo studio di pittori quali: Federico Barocci (pittori di origini urbinate importante esponente del Manierismo che ora è considerato uno dei precursori del Barocco) e Claudio Ridolfi (pittore italiano che si stabilì a Corinaldo). Infine Cantarini "detto il Pesarese"si recò alla scuola di Guido Reni a Bologna del quale aveva già potuto ammirare quì a Pesaro la Pala d'altare della Madonna con Bambino in gloria, San Tommaso e San Girolamo presente all'interno del Duomo che ora si trovano presso i Musei Vaticani. Simone venne definito un'artista eclettico e inquieto, morì a soli 37 anni ma ci lasciò opere mirabili e dense di dolcezza.\\
		Giannandrea Lazzarini lo ricorda così:"Quand'ei pingea nel bel lavor sovrano, le grazie scendea a reggergli la mano", sono infatti grazia e compostezza insieme al tema della penitenza che era tipico del 600 controriformato a caratterizzare lo stile classicista del pittore pesarese.
		
		\section{Sala Sacro e Profano}
		Proseguendo oltre nella visita entriamo in questa nuova sala.
		
		\subsection{La Trinità con Madonna di Giannandrea Lazzarini}
		L'imponente opera raffigura la Vergine sospesa tra le nuvole che intercede con la Trinità per gli uomini tutti, in cambio del sacrificio di Cristo in modo da cancellare il peccato commesso dai progenitori Adamo ed Eva (rappresentati in basso), era stata realizzata nel 1759 per l'altare della chiesa Santissima Trinità dell'Ospedale. L'opera gode di ampio respiro (ariosità) e vivacità dei colori. Un caso emblematico fu quello del pesarese Giannandrea Lazzarini artista di carattere eclettico che fu: pittore, architetto, teorico, critico d'arte, poeta e teologo.\\
		Giannandrea Lazzarini progettò infatti quale architetto Palazzo Mazzolari e Palazzo Olivieri (ora sede del Conservatorio di musica Gioacchino Rossini). La formazione del poliedrico artista avvenne presso il vadese Francesco Mancini dove apprese un tipo di istruzione romano-emiliana di matrice classicista i cui studi vertivano su: Raffaello, Carracci e Barocci. Nel 700 si dibatteva spesso in ambito del pesarese sull'usanza dei pittori contemporanei di ispirarsi all'arte antica e Giannandrea curioso assisteva a questi dibattiti culturali.
		
		\subsection{Il Mercato di Aureliano Milani}
		Questo quadro raffigura una vivace scena di cronaca popolare. Lo sfondo che fà quasi da protagonista è la Piazza romana della Bocca della Verità in cui si muovono figure delle più disparate fascie della società: religiosi, nobili, umili lavoratori, ambulanti, cavadenti, venditori di vasellame e un'Arlecchino. Posto in alto è inoltre curioso notare un piccolo teatro di burattini.
		
		\section{Sala Natura e Inganno}
		
		\subsection{Ceramiche con decoro alla Rosa di Pesaro}
		Tornando indietro ci accorgiamo di una grande teca in vetro sulla sinistra al cui interno sono conservate le pregiate e tipiche ceramiche pesaresi con decoro alla Rosa di Pesaro, ma facciamo un passo indietro per ripercorrere la storia di queste opere ceramiche.\\
		Il 600 si apre con la devoluzione del ducato dei Della Rovere allo Stato Pontificio questo sarà un secolo buio, infatti la crisi economica si rifletterà nella produzione delle ceramiche che sarà quasi totalmente assente. Nel 700 grazie a una ripresa culturale ed economica rifiorì la maiolica locale, il decoro floreale denuncia un ritorno alla natura in linea con i valori dell'epoca. In posizione centrale all'interno della teca si può infatti ammirare uno splendido esempio di vassoio ovale con la tipica decorazione con la Rosa di Pesaro, che è parte di un ricco servizio. La Rosa viene rappresentata a lato e mai al centro, in maniera florida e tondeggiante. La tipica colorazione brillante bordeaux si otteneva tramite la pittura con smalti e ben tre cotture della ceramica in cui veniva applicata. Il vassoio alla Rosa di Pesaro che potete vedere venne prodotto dalla fabbrica Casali e Callegari di Pesaro, la più importante delle botteghe ceramiche di Pesaro del diciottesimo secolo, questa decorazione venne poi diffusa da Giannandrea Lazzarini. Un altro motivo decorativo tipico dell'epoca era il Ticchio, termine dialettale che indicava un arbusto filiforme e serpentinato che veniva rappresentato contornato da tre crisantemi e sostenuto da due peonie, potete ammirare sempre all'interno della teca in basso a sinistra una splendida caffettiera della fabbrica Casali e Callegari che è ornata con questo motivo.
		
		\subsection{Fiori, Frutta e un vassoio di calici di vetro di Christian Berentz}
		Alle vostre spalle rispetto alla teca contenente le ceramiche proseguiamo ora la visita per ammirare questi splendidi esempi di Nature Morte.\\
		Tra le varie di questa sala spicca in particolar modo quella realizzata da Christian Berentz. La frutta dell'autunno, le foglie e i fiori si dispongono in una diagonale ascendente che culmina con la luminosa apertura del cielo, al cui chiarore si contrappone il pesante pilastro in penombra la cui presenza viene alleggerita dalla grazia di due farfalle bianche. In posizione centrale risalta un vassoio con al di sopra due calici, una tazza e un boccale. La luce bagna i petali di rosa, le gocce d'acqua rinfrescano i fichi, gli acini dell'uva, il melograno e la ruvida scorza della  zucca. La luce si riflette inoltre sul metallo del vassoio, del coltello e il vetro dei calici, restituendo vivacità al fluido liquore al loro interno e risaltandone le bollicine del vino. Tanta esuberanza decorativa lascia intuire il messaggio della caducità della vita a cui è impossibile sfuggire, ciò è evidenziato dal dettaglio dei petali che cadono sparsi per via dell'appassire della rosa, la pesca bacata, la frutta in parte mangiata ed il dettaglio delle farfalle di cui è risaputo il breve ciclo vitale. Il pittore Cristian Berentz è originario di Amburgo e coniuga l'opulenza tipica della sua formazione romana, con un'accurata descrizione degli oggetti e inserti vegetali tipica della pittura nordica.
		
		\subsection{Tromple l'oeil con Sonetto di Antonio Gianlisi Junior}
		Proseguendo in avanti sulla sinistra possiamo trovare questo particolare tipo di opere.\\
		Gli oggetti come potete vedere sono distribuiti su uno sfondo che simula delle pareti in legno, per via dell'illusionismo ottico.\\
		Vogliamo però soffermarci ad osservare sull'opera all'estrema sinistra, il cui soggetto dello scheletro con falce ci fa meditare con un sonetto sulla caducità della vita sottolineata dai particolari degli occhiali e della clessidra. Vita che deve essere affrontata attraverso la preghiera come si può notare dai particolari dei rosari, croce e libri sacri ma anche attraverso l'esercizio della scrittura i cui oggetti vengono anch'essi rappresentati in preparazione poi alla morte. Il pittore di quest'opera fu Antonio Gianlisi Junior originario ed attivo a Cremona nella seconda metà del 700.
		
		\section{Sonosfera}
		Unica al mondo, all'interno dei Musei Civici è presente questa particolare struttura per l'ascolto profondo a forma di geoide progettata per essere acusticamente perfetta, isolata dall'esterno e completamente fono-assorbente all'interno. La struttura internamente ricorda un anfiteatro, nel quale il pubblico si siede in una doppia cavea per poter attraverso l'ascolto partecipare alla comprensione dei due spettacoli realizzati appositamente per questa struttura.
		
		\subsection{Video 1: Frammenti di Estinzione nell'Orologio Climatico}
		Nato per il progetto "Fragments of Extinction-Il Patrimonio Sonoro degli Ecosistemi", è stato il primo dei due video ad oggi a essere stati pensati per essere degli spettacoli all'interno di Sonosfera. Il video nel 2014 ha ottenuto il brevetto internazionale ad opera dell'ideatore David Monacchi, professore del Conservatorio Rossini.\\
		Il progetto studia e propone i paesaggi sonori delle foreste primarie equatoriali registrati da Monacchi stesso durante le sue spedizioni in tutto il pianeta. Visti i drastici mutamenti del clima e degli habitat naturali a livello globale questa esperienza video vuole convogliare l'attenzione dello spettatore sul pericolo della "sesta estinzione di massa" che sta avvenendo proprio ora e far sì che il pubblico diventi più consapevole di questo fenomeno, che porti a un più veloce transizione ecologica e che muti anche la percezione dello spettatore portandolo ad essere più rispettoso nei confronti della Natura.Questa esperienza si collega con la visione all'esterno sulla facciata del Municipio dell'Orologio Climatico dove si può vedere il tempo a disposizione dell'umanità per esaurire il "budget" di CO2 da emettere in atmosfera prima di oltrepassare la soglia critica (+1,5 gradi) e della quota di energia prodotta da fonti rinnovabili.
		
		\subsection{Video 2: Raffaello in Sonosfera}
		Questa seconda esperienza è dedicata a Raffaello ed è stata messa a punto per il quinto centenario della morte del Divin Pittore. Questo video ci porta attraverso l'ascolto di musica riascimentale (che proviene dal disco Raphael Urbinas di Simone Sorini Syrenarum) tratta da nientemeno che un sonetto di Raffaello stesso posto in musica secondo una consolidata prassi dell'epoca si ha quindi la possibilità e il piacere inusuale di  visionare gli affreschi di Raffaello veicolati con uno splendido canto e melodia coeva, che verrà assimilato alla figura di Apollo nel relativo riquadro.  Attraverso una visione a 360 gradi dell'operato del pittore, nella Stanza della Segnatura dei Musei Vaticani e andremo a comprendere un nuovo modo di vedere queste spettacolari opere, in cui ci verranno date delle chiavi di lettura innovative alla comprensione di esse.\\
		La Sonosfera è infatti ad oggi lo spazio migliore per condurre un'indagine esplorativa sull'arte a vari livelli e così come Raffaello 500 anni fa seppe concepire e dar vita ad un tale livello immersivo nella sua opera pittorica, allo stesso modo noi ci siamo proposti di rivelarne attraverso un'indagine metodologica ed estetica i canoni compositivi da lui magistralmente esposti, a partire dalla decifrazione dei simboli, l'esplorazione dei modelli prospettici, il controllo della spazialità, sino all'apparizione dell'umano e del finale trionfo del divino.\\
		La visione in Sonosfera è inoltre in grado di rendere palesi e chiarire il significato di alcuni dettagli che si farebbe difficoltà a notare persino davanti all'originale stesso.\\
		La visione in Sonosfera  nella prima parte utilizza un effetto sinopia (disegno preparatorio che facevano i pittori con la matita sanguigna) per mettere in risalto e ove possibile sciogliere il complesso sistema di simboli collocati da Raffaello magistralmente sempre nei punti più strategici delle sue opere.\\
		Un dettaglio che verrà analizzato per esempio è il Timeo (libro) in mano a Platone, che riassume il sapere filosofico raggiunto nell'antica Grecia ed è posizionato nel punto di fuga della Scuola di Atene e si trova all'opposto rispetto al punto di fuga dell'affresco di fronte dove si trova invece l'ostia. La fede e la ragione si fronteggiano, così come la poesia e la giustizia negli altri due affreschi più piccoli.\\
		Un dettaglio che l'esperienza in Sonosfera mette in luce è quello della Tetraktys, Sacra Decade per i Pitagorici, successione aritmetica e simmetria geometrica dei primi quattro numeri naturali. La Tetraktys genera quindi i rapporti che costituiscono la base dalla quale nascono gli intervalli musicali sia frequenziali che reali secondo la suddivisione della corda, intervalli che diventeranno la base dell'intero sistema musicale occidentale che è inoltre uno dei motivi per i quali tra le tante opere è stata scelta questa.\\
		Vedrete poi cari spettatori che anche gli antichi strumenti greci dipinti nel Parnaso si animeranno e ne ascolterete i suoni che nell'ambiente pastorale e Neo-Platonico del Parnaso si mescoleranno a quelli naturali che fanno parte della percezione sonora del paesaggio dell'opera. La terza ed ultima parte sorregge con musica sacra cinquecentesca l'affresco sulla teologia La Disputa del Sacramento, il quale viene esploso in una sorta di immersione nell'opera stessa.
		
		\section{Fonti}
		Le informazioni riportate sono tratte dalle audio-guide dei Musei Civici.
		
		\vspace*{\fill}
		% Print license shield
		\doclicenseThis
	\end{flushleft}
\end{document}
